%Chapter 1

\renewcommand{\thechapter}{1}

\chapter{Introduction}

\section{Extragalctic Astronomy and the AGN Paradigm}
Nebulae, which stand for clouds in Latin, have been an integral part of astronomy since the middle ages and were one of the main driving forces behind extragalactic hypotheses. Their first known record dates back to 150 AD, who recorded 5 objects in the sky as ``nebulous'' in his books VII-VIII  of \textit{Almagest}. Abd al-Rahman al-Sufi noted a ``little cloud'' during 964 AD in his work \textit{Book of fixed Stars}, which we now know as the Andromeda galaxy. On July 4th 1054, Chinese and Persian astronomers have recorded the supernova event of the Crab nebula.

With the advent of refracting telescopes, mainly built based on the designs developed by Hans Lippershey in the early 17th century, Orion neubula has been detected and studied in detail by French and Swiss astronomers. By the middle of the 18th century several more nebulae have been cataloged. In 1715, Edmund Halley published a list of six, Jean-Philippe de Cheseaux twenty with eight newer ones in 1745,Nicolas Louis de Lacaille 42 between 1751-1753. Later in 1781, Charles Messier compiled a list of 103 objects, which are now called as Messier objects, including the famous M87.

The number of known objects have substantially improved with works of William Hershcel and his sister Caroline who published catalogs with more than a thousand nebulae during 1780s. However, it was unclear at the time whether some of these objects were located outside the Milky Way. 
