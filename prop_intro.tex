\section{Introduction}

%what is the broad area that the research problem is in?

%--introduce AGN with some history
%--introduce the notion of SMBH
%--introduce the jets
%--what are they? what are they capable of doing? how do they do it? what are its implications on a broad scale? why should we study them?


By the middle of the 18th century, astronomers cataloged many nebulae. %, which are bight dust and gas clouds.
However, it was uncertain whether some of them lived outside the Milky Way.  Thomas Wright hypothesizes that some of these objects are galaxies similar to our Milky Way. Immanuel Kant misinterpreted this hypothesis and called them ``island universes." By the late 18th century, William Herschel observed more than 5000 nebulae.  Later, upgraded telescopes found many point sources present within them. \citet{fath1909spectra} found that some nebulae emit a continuum--similar to a  collection of stars--with stellar absorption lines.  Interestingly, some emission lines from one nebula, NGC 1068, were found to have a width of  \siml3000 km/s. 
% It displays the characteristic of both galactic and extragalactic origins.
\citet{slipher1913radial} noted that the lines from another nebula, Andromeda, display redshift that are higher than the escape speed of Milky way and are indicative of an extragalactic origin. \citet{curtis1919modern} adds many other arguments and concludes that  Andromeda might be an independent galaxy by itself. \citet{opik1922estimate}  estimated the distance of Andromeda to be about 450 kpc (which is half the actual value), which indicated a location outside our Galaxy. Soon, \citet{hubble1929relation} settled the debate by using Cepheid variables. Cephids are stars with a cycle of brightness whose frequency correlates with their luminosity. This correlation allows determining the distance to them. Using Cepheid variables in Andromeda and other objects, Hubble computed their distance and confirms their extragalactic origin.

The nature of the bright emission lines, however,  remained unknown for a decade or two.  \citet{seyfert1943nuclear} obtained the spectra of six galaxies and found that they show high-excitation emission lines superimposed on a stellar-like continuum. He also found that some galaxies show broad emission lines while some others narrow. \citet{woltjer1960magnetostatic} noted that parsec scale emission from these galaxies requires a mass of \siml$10^8~M_\odot$ where $M_\odot$ denotes the mass of the Sun. \citet{fowler1963star} made an important advance by suggesting that the centers of these galaxies host a massive star; they would emit by accreting dust from the surroundings. A year later, another idea emerged, which assumed that the central object is a black hole rather than a massive star \citep{salpeter1964accretion,zel1964estimating}. Such galaxies whose bright centers are powered by accretion on to a black hole are what we now call as Active Galactic Nuclei (AGN). 
AGN are the most luminous and steady sources of luminosity in the universe. They have a broad spectrum covering 20 orders of magnitude in frequency with luminosity ranging from $10^{40}-10^{49}$ ergs s$^{-1}$ and are one of the most actively studied class of objects in astronomy.

Radio astronomy also provided important contributions towards our understanding of AGN. Its history dates back to the early 1930s. In 1932, Karl Guthe Jansky, an engineer at Bell Labs, noted that a radio signal peaked with a period of 24hrs and suspected that they were radio waves from the Sun. Upon gathering more data, he found that the signal repeated with the frequency of a Sidereal day: the time it takes for distant stars to rotate once in our field of view. These finding eventually lead Jansky to conclude that the signal came from the central regions of our galaxy \citep{jansky1932directional}, and it marked the first serendipitous discovery of an astronomical radio source and the birth of radio astronomy. He was honored for his seminal contributions to this field by naming the units of flux as Jansky (Jy) after his name. Inspired by this discovery, Grote Reber, an amateur radio engineer, built a parabolic radio telescope with 9m diameter in his backyard. He mapped the emission from the galactic center at 160 MHz and not only confirmed Jansky's discovery \citep{reber1940cosmic} but also showed that the emission exhibits a non-thermal spectrum. Although World War II intervened during this period and hindered research in astronomy, it led to advances in radar technology and consequently, in radio astronomy. 

The development of Earth-rotation aperture synthesis techniques by Martin Ryle and Antony Hewish in the 1950s, and the arrival of efficient computing machines capable of performing the required inverse Fourier transforms made it possible to build radio telescopes with one mile alter with 5 km apertures. These telescopes 
%The brightness temperatures of some sources are greater than $10^6 K$ and indicate a non-thermal origin \citep[e.g.,][]{bolton1948variable}. 
%\citet{1950AuSRA...3..234S} located a bright radio source that coincided with the location of M87, which we now identify as a nearby AGN. Later, \citet{1954ApJ...119..215B} identified extended optical emission from M87.
were used to conduct the 3C and 3CR surveys at 159 MHz and 178 MHz, respectively, detecting several hundreds of radio sources. However, their low positional accuracy precluded accurate identification of their optical counterparts. 
Later, \citet{schmidt19633c} identified a star in optical wavelength with a magnitude of 13; it also contained a structure that he refers to as ``a faint wisp or jet." They both were close to the radio positions of two bright components in 3C 273. The optical spectrum of  3C 273 exhibits a complex continuum with broad emission lines \citep{oke1963absolute}.
\citet{schmidt19633c} shows that the emission lines from 3C 273 match Balmer and Mg II lines at a redshift of $z=0.158$. This redshift implies an extragalactic origin and a higher than typical luminosity. Astronomers started to believe that such sources are some ``radio stars" and started to call them \textit{quasi stellar sources} or \textit{quasars} \citep{chiu1964gravitational}. After \citet{salpeter1964accretion} and \citet{zel1964estimating} advanced the idea of a black hole at the center of an AGN, \citet{lynden1969galactic} put forward an idea that accretion on to blackholes with sufficient mass can power the high luminosity of quasars. He also argued that it can produce the observed thermal continuum and broad emission lines and can be the process behind quasars and AGN. This idea was widely accepted by the community, and it marked the dawn of the basic AGN paradigm.
\citet{hogg1969synthesis} found extended radio emission from M87 that coincides with its extended optical component. Soon high-resolution observations by the radio interferometers at Cambridge revealed extended emission from many sources \citep[e.g.,][]{1973MNRAS.165..369N,turland19753c}. The extended emission components come from sources with a wide range of radio powers. Finally, it is the Very Large Array (VLA) that revealed the ubiquitous nature of this extended emission. They appear in all forms of AGN, which emit bright radio emission.
These extended emission components are what we now refer to as ``relativistic jets from AGN," and they form the broad subject of my work. 
 %Many jets display localized brightness enhancements called knots in some cases and hotspots in some jets.  The knots are localized sites  which accelerate electrons perhaps via internal shocks to produce enhanced emission \citep[e.g.,][]{komissarov1998large,1988ApJ...334..539D,2015ApJ...809...38M}. The hotspots are the regions where the jet terminates after slamming in to the inter galactic medium (IGM)
 Fig.~\ref{fig:M87_showpiece} shows a VLA 8.4 GHz radio map of M87 at 0.2\as~resolution as an example.
\begin{figure*}
    \gridline{
    \boxedfig{images/misc/M87_proposal_showpiece.eps}{0.849\textwidth}{}
    }
    \caption{The 8.4GHz radio image of the relativistic jet in M87    \label{fig:M87_showpiece} }
\end{figure*}

It is now widely accepted that practically all the galaxies host a supermassive black hole (SMBH) at their center \citep[e.g.,][]{richstone1998supermassive} and accretion on to SMBHs powers all forms of AGN. About 10\% of the AGN produce collimated bipolar relativistic jets of radio-emitting plasma \citep[e.g.,][]{Padovani_2017}; they transport energy and momentum from the central parsec scale of the AGN out to kilo-parsec (kpc) and ofttimes to Mpc scales. It is plausible that a strong magnetization of the accretion disk couples to the black hole's spin and produces a high latitude outflow at relativistic speeds \citep[e.g.,][]{meier2001magnetohydrodynamic}. They produce a broad spectrum ranging from radio to X-rays \citep[e.g.,][]{harris2002x}.

The kinetic and radiative power of the jets can stimulate and limit the growth of galaxies;  the jets may also be responsible for the high energy cosmic rays and the intergalactic magnetic fields \citep[see ][]{blandford2019relativistic}. It is plausible that the jets may have maintained the black hole masses of the early universe \citep{churazov2005supermassive} via feedback. There is also growing evidence that the output of the jets may have shaped the baryonic (i.e., protons, neutrons and alike) part of the local universe  \citep[see][]{fabian2012observational}. If so, the jets have profound implications for the evolution for the universe.

Despite understanding the significant role the jets can play, we only know a little about the jet physics even after studying them for more than half a century. Their particle composition--electrons and positrons or electrons and protons---is uncertain as is the speed of the jets on kpc to Mpc scales. Importantly, for the present work,  how large-scale ($>$100 kilo parsec) jets emit X-rays hundreds of kpc away from the AGN is unclear while competing models imply orders of magnitude differences in their jet powers. My current and proposed work is an attempt at elucidating the properties of X-ray emission from such jets.
%and sometimes even up to GeV \citep[e.g.,][]{Meyer_2019}. The jets can heat their host galaxies and also the intergalactic medium \citep[e.g.,][]{broderick2012cosmological,2015IAUS..313..101B,2016MNRAS.461..967M}.  They can also produce high energy cosmic rays and intergalactic magnetic fields \citep{blandford2019relativistic}. Such feedback can regulate the formation of stars and galaxies \citep[e.g.,][]{schawinski2007observational}. If so, they have important implications for the evolution of the universe \citep{blandford2019relativistic}. However,  despite studying them for more than half a century, we still know little about the physics of jets. The exact particle composition of the jet (electron-positron or electron-proton) is unclear, and the mechanism by which the jets produce X-rays is still under debate. These two factors, combined,  imply diverging powers of the jet, which translates to enormous differences in their impact on the evolution of the universe. My work focuses on understanding the nature of X-ray emission from the large jets.

The rest of this document is organized as follows. Section \ref{sec:agn_current_model} provides a brief overview on the current model of AGN. Section \ref{sec:taxonomy}    describes the taxonomy and unification of AGN.  Section \ref{sec:engine} discusses the central engine of the AGN. Section \ref{sec:radative_processes} presents a summary of radiation mechanisms and relativistic effects in the context of a jet. Section  \ref{sec:xray_jets} describes X-ray jets and their properties. Section \ref{sec:proposal} outlines the proposed research and section \ref{sec:preliminary_results} provides the preliminary results of this work.
\subsection{Active Galactic Nuclei: The current model \label{sec:agn_current_model}}
\begin{figure*}
    \gridline{
    \boxedfig{images/misc/agn_model.jpg}{0.849\textwidth}{}
    }
    \caption{A simplified schema of the current model of the AGN, not drawn to scale. The taxonomy of the AGN is a result of the presence or absence of various components and observational effects. Figure adapted from \citet{beckmann2012agn}.    \label{fig:agn_model} }
\end{figure*}
The idea of a black hole at the center of AGN is a groundbreaking one as it explains puzzling  features of the AGN such as short variability time scales and extremely high luminosity. Fig~\ref{fig:agn_model} shows a simplified view of the currently accepted model of AGN.  A SMBH at the center forms the \textit{central engine} that is responsible for all the AGN activity. It accretes matter via an accretion disk. For powerful systems, the spectrum of the disk peaks in the optical/UV and is traditionally called the ``big blue bump." The accretion disk photoionizes the neighboring high density and dust-free gas clouds, which leads to the production of strong  emission lines. These clouds mostly occur within a parsec of the SMBH \citep{peterson2006broad}; they move with roughly keplerian speeds, and, as a result,  broaden the emission lines. Hence, this region is known as the broad-line region (BLR). The low-density, low velocity, ionized gas, a few hundred to thousands of parsecs away from the black hole, move with slower speeds, and produce narrower emission lines. This region constitutes the narrow-line region (NLR). A dusty torus structure may also surround the black hole-disk system. When viewed edge-on, this torus obscures the emission from the BLR. Such sources lack broad emission lines and fall under the Type II class of AGN. If the AGN are observed face-on, the broad emission lines are visible, and such sources fall under the Type I AGN class. This viewing angle based classification plays an important role in understanding the AGN, although it is clear that the torus is likely more complex and ``clumpy" that what is shown here \citep[see][and references there in]{H_nig_2019}.
 \subsection{AGN Taxonomy \label{sec:taxonomy}}
AGN were classified, traditionally, as radio-loud and radio-quiet, in an empirical sense. The dividing factor is the ratio of radio to optical flux density ($R$). Those with $R\leq10$ are classified radio-quiet and those above the limit radio-loud. However, this distinction is not just taxonomic but an intrinsic one. It stems from the presence or absence of a jet \citep{Padovani_2017}. Hence, in what follows, they would instead be treated as non-jetted and jetted AGN.
\subsubsection{Jetted AGN}
\textit{Radio Galaxies} (RG) are the initial detections of jetted AGN with identified optical counterparts \citep[e.g.,][]{1954ApJ...119..215B}. They are AGN with jets that are aligned away from our line of sight. RGs are further classified in to two types based on their optical spectra: broad line (BLRG), which display both broad and narrow emission lines, and narrow line (NLRG), which only display narrow lines  (see Fig~\ref{fig:agn_model}).  \citet{fanaroff1974morphology} found that the morphology of the large-scale jet depends on its low-frequency radio luminosity. In the lower power Fanaroff-Riley Class I (FR I), the jets display a plumy structure which fade away with distance from the core. In contrast, the FR-II type jets are collimated and travel long distances before they ram in to the inter-galactic medium terminating in a \textit{hotspot}. Most of the sources with luminosity at 1.4 GHz $L_{1.4GHz}<5\times10^{25}~\text{W Hz}^{-1}$~
%[UNITS. ALSO IT IS BEST TO MENTION THE 178 MHZ AT WHICH FR DO THE SEPARATION, AS IT IS LESS CONTAMINATED BY THE BEAMED CORE]
belonged to the FR I class while most of those above the limit to the FR II class. Note that \citet{fanaroff1974morphology} used the luminosity at 178 MHz to separate the FR-I and FR-II classes as there will be minimal contribution from the beamed core at this frequency.  Fig.~\ref{fig:FR_I_II} shows an example image for each class.
\begin{figure*}
    \gridline{
    \boxedfig{images/misc/3C272.1_FR_I_example}{0.35\textwidth}{(a) FR I}
    \boxedfig{images/misc/CygA_FR_II_example}{0.607\textwidth}{(b) FR II}
    }
    \caption{Typical example images for the FR I and II classes. (a) shows an FR-I source, 3C272.1, image taken from DRAGN (http://www.jb.man.ac.uk/atlas/). The brightness of the jet gradually decreases away from the central nucleus (\textit{the core}). (b) shows an FR-II source, Cygnus A, taken from NED (http://ned.ipac.caltech.edu). Two bright hotspots and associated radio lobes lie on either side of the core that are connected by an otherwise invisible jet.\label{fig:FR_I_II} }
\end{figure*}
\textit{Blazars} form another prominent class of jetted AGN in a sense the compliment to radio galaxies, as they are viewed with a small angle. They are further divided into two classes: flat-spectrum radio quasars (FSRQs), which display broad emission lines and BL Lacertae objects (BL Lacs), which have featureless spectra. Under the primary unification scheme 
\citep[e.g.,][]{urry1995unified}, FSRQs and BL Lacs are the closely aligned counterparts of FR-II type jets and FR-I type jets, respectively. Blazars have the highest apparent luminosity of all AGN families where the bulk relativistic motion causes enhancements in the apparent luminosity of up to $10^9$, a phenomenon known as \textit{relativistic beaming} (see section \ref{subsec:relativistic_effects}). The sources that are intermediate between radio galaxies and blazars are called Steep Spectrum Radio Quasars (SSRQs). The radio spectrum of the SSRQs is steeper than that of blazars, as there is a contribution from the lobe, which has a steeper spectrum, as opposed to the core, which has a flatter spectrum. Because of a major contribution from the beamed emission from the core, blazars appear as ``core dominated" while emission from the lobes constitute a major contributor to the observed emission in radio Galaxies, and hence they appear as ``lobe-dominated" \citep{1997iagn.book.....P}.
%Typically, the blazar category sources are observed to be ``core dominated"; the bulk of their radio emission comes from the relativistically beamed core. In contrast, the radio galaxies are observed to be ``lobe dominated"; the bulk of their radio emission comes from the lobes, which are not beamed. As the blazar jets are closely aligned, the beamed emission from their core overwhelms the emission from the lobe. Because the jets are aligned away from us in the radio galaxies, the emission from their lobes surpasses the de-beamed emission from their core.



\subsubsection{Non-jetted AGN}
These are the most common class of AGNs in the local universe, and Seyfert galaxies \citep{seyfert1943nuclear} are the first ones to be identified in this class. Typically, the Seyferts differ from other normal galaxies by their highly ionized emission lines and a bright, central core. Similar to the jetted AGNs, the Seyfert galaxies are divided into two classes based on their optical spectra \citep{khachikian1974atlas}: Seyfert I type, which display both narrow and broad emission lines, and Seyfert II type, which only display narrow emission lines (see Fig.~\ref{fig:agn_model}). Apart from the optical classification, the Sefyferts are also classified based on their X-ray emission, which relies on the intrinsic absorption in the soft X-ray band ($E\ll 5~\text{keV}$). A hydrogen absorption column density of $N_H=10^{22}~\text{cm}^{-2}$ draws the line between the type I and type II Seyferts. The ones above the threshold are strongly absorbed and fall into the type I class, while the ones below have lower absorption and fall into the type II class. 

There may also be sources with a low luminosity that go undetected in the all-sky surveys. For example, the bolometric luminosity of Sagittarius A*, which is the black hole at the center of our Milky Way galaxy, is less than $L_{bol}=10^{37}~\text{erg sec}^{-1}$. Such emission would not be detected if it were to lie in a distant galaxy. It is believed that these sources may have experienced some activity in the past and are presently undergoing a quiescent phase. Quasi Stellar Objects (QSO), or radio-quiet quasars are another type of non-jetted AGN. Note that, in the jet community, quasars refer to jetted-AGN with closely aligned jets, and unless otherwise stated a QSO refers to a non-jetted AGN. Similar to the Seyferts, the QSOs are further divided in to type I and type II, which display broad lines and only narrow lines, respectively. Unlike Seyferts, which often appear in spiral galaxies, the QSOs are generally seen in the elliptical ones.

\subsection{Radiative processes \label{sec:radative_processes}}
The radiation from the jets exhibits a broad spectrum. The core, which dominates the SED when aligned, ranges from the radio to TeV \citep{harris2006x}. Synchrotron radiation and possibly Compton upscattering are two of the important processes involved in the production of these broadband spectra. Moreover, this emission can be enhanced or diminished by relativistic beaming based on their alignment with our line of sight. In the large-scale jet, the high energy mechanism is unkown. This section provides an overview of the two main radiation processes and the involved relativistic effects.


\subsubsection{Synchrotron Radiation}
One of the most important characteristics of synchrotron radiation is a high degree of polarization.  This characteristic led to the first-ever proof that synchrotron radiation produces the optical emission from the jet of M87 \citep{baade1956polarization}. It also quickly explained the radio emission from other jets.

 Charged particles, when accelerated, emit electromagnetic radiation  \cite[for a review, see][]{longair_2011}. If the charged particles are relativistic and a magnetic field accelerates them, they produce synchrotron radiation. The charged particles change direction because the magnetic field exerts a force perpendicular to the original direction of the motion.  For a charged particle (mass=$m$ and charge$=Z$) that moves with a speed $v$ in a magnetic field $B$, its average radiated synchrotron power is proportional to 
%energy $E=\gamma mc^2$ (where $\gamma$ is the Lorentz factor) and speed v 
\begin{equation}
    \langle  P_{synch} \rangle\propto \frac{Z^4\gamma^2B^2v^2}{m^2}
\end{equation}
 where $\gamma$ is the Lorentz factor.  Because $P_{synch}\proptom^{-2}$, the emission is extremely efficient for electrons and positrons when compared to protons that have a mass ratio of \siml 2000. For an electron and proton travelling with the same speed, the power radiated by a proton will be smaller by a factor of $3\times10^{-7}$ compared to an electron
 %[OF THE SAME ENERGY OR SAME GAMMA?]
 . That means, for a given synchrotron luminosity, a jet with radiating protons would demand a significantly higher energy budget than an electron-jet. This difference plays an important role in the studies that aim to determine the particle composition of a jet. For an electron (or a positron), the power is
\begin{equation}\label{eq:synch_pow}
    \langle  P_{synch} \rangle=\frac{4}{3}\sigma_T\beta^2\gamma^2cU_B
\end{equation}
where $\sigma_T$ is the Thomson scattering cross-section, $c$ is the speed of light, $\beta=v/c$ and $U_B=B^2/8\pi$ is the magnetic field energy density. The specific luminosity of a electron peaks at $0.29\nu_{crit}$, where $\nu_{crit}$ is the critical frequency that is given by:
%An electron emits approximately half of the total power [WITHIN WHAT BAND PASS? MAYBE YOU WANT TO SAY THAT THE SPECIFIC LUMINOSITY EMITTED PEAKS AT 0.29 NUCRIT] at a \textit{critical frequency}, which is given by:
\begin{equation}
    \nu_c = \frac{3\gamma^2eB}{4\pi m_e c}\sin{\theta}
\end{equation}
where $e$ is the electron charge, and $\theta$ is the angle between the magnetic field and the velocity of electron. For an electron to emit at $\nu\sim1$ GHz in a magnetic field of $B\sim10^{-6}$ G, it requires $\gamma\sim10^{4}$. Such extremely relativistic electrons in a jet can only be produced by an efficient particle acceleration mechanism, such as the first-order Fermi acceleration \citep{1949PhRv...75.1169F}.

The radio emission from a jet usually takes a powerlaw form $F_\nu\propto \nu^{-\alpha}$ where $F_\nu$ is the flux density $\nu$ is the frequency, and $\alpha$ is the spectral index. Lower $\alpha$ indicates a harder/flatter spectrum, while higher $\alpha$ indicates a softer/steeper spectrum. For a group of synchrotron emitting electrons in a jet, the resulting spectrum is a superposition of the individual spectra of the electrons. If the electron energies follow a powerlaw distribution, $n(E)dE=E^{-s}dE$ (where $n(E)dE$ in the number density of electrons in the energy range $E \text{~to~} E+dE$), the final spectrum is also a powerlaw given by $F_\nu\propto \nu^{-\alpha}$, where $\alpha=(s-1)/2$. In many synchrotron sources, $\alpha\approx0.75$ at $\nu\approx1$~GHz implying that $s\approx2.5$. The value of $s$ indicates the index at which the electrons are originally injected in to the jet possibly via shocks and get modified by synchrotron losses over time. For example, the power emitted by electrons is proportional to $\gamma^2$, so that higher-energy electrons radiatively cool much faster. Moreover, as the critical frequency is also proportional to $\gamma^2$, the spectra will become steeper at higher frequencies with time since the acceleration event. This \textit{spectral aeging} can be used as a tool to infer the age of the jet \citep[e.g.,][]{Harwood_2015}: for a jet that is injecting electrons into the lobes, for example from a hotspot, the spectrum at the injection point is flatter than at the points away from it, and the differences in the spectral indices can be used to measure the age of the jet. 


At lower synchrotron frequencies, the radiating plasma  become optically thick and absorbs the synchrotron radiation it produces. The resulting spectrum at lower frequencies is also a power law but with a steeper slope of $-5/2$. This process is known as \textit{synchrotron self-absorption} (SSA).

It is important to note that the observed synchrotron power depends on both the electron Lorentz factor and the magnetic field (Eq.~\ref{eq:synch_pow}), which precludes their independent estimations from the observations. Moreover, the uncertainty in the particle composition of the jet presents another problem in determining these parameters. If the ion/electron energy density ratio is $\eta$, then the total energy density of the particles would be $(1+\eta)U_e$, where $U_e$ is the energy density of the electrons. To minimize the total energy density, the ratio of energy densities of the particles to that of the magnetic field can be shown to be \citep{longair_2011}:
\begin{equation}
    \frac{(1+\eta)U_e}{U_B}=\frac{4}{3}
\end{equation}
which is approximately equal to 1. Such a condition is called as \textit{equipartition}, and is widely adopted, although it is not supported by any micro-physics mechanism that would drive the plasma to equipartition.


\subsubsection{Inverse Compton Radiation\label{subsec:ic}}
Apart from synchrotron radiation, the emission from electrons scattering ambient photons can also be a significant contributor to the observed emission from jets.
When a photon transfers its energy to a non-relativistic electron, it is called Compton scattering. However, if the electron moves at relativistic speeds, it can scatter a photon from a lower frequency into a higher frequency. This process is called \textit{inverse Compton (IC)} scattering. The inverse Compton radiation bears the same polarization as the radiation field that is being up scattered \citep{uchiyama2007infrared}. The total IC power radiated by a single electron by up scattering a radiation field of energy density $U_{rad}$ is given by
\begin{equation}
    P_{IC} =\frac{4}{3}\sigma_T\beta^2\gamma^2cU_{rad}
\end{equation}
The ratio of the synchrotron to IC power would then become $P_{synch}/P_{IC}=U_B/U_{rad}$. The means, although both the synchrotron and IC processes are inevitable in a jet, it is the relative energy density of the magnetic field to the radiation field that determines the dominant emission mechanism. 

For a powerlaw distribution of electron energies with an index $s$, the IC spectrum also follows a powerlaw with an index of $(s-1)/2$, which is the same as the index for synchrotron emission. When the synchrotron electrons themselves upscatter the radiation that they produce, it leads to the \textit{synchrotron self Compton (SSC)} radiation. The SSC emission can also contribute back to the radiation field, leading to multiple SSC scatterings. Although it can amplify the emission, once the brightness temperature exceeds $T_b\sim10^{12}$~K, synchrotron self-absorption dominates and sharply cools off the electrons.

\subsubsection{Relativistic effects\label{subsec:relativistic_effects}}
It is now widely known that the jets from AGN are relativistic: their speeds are close to the speed of light. The jets are characterized by their bulk Lorentz factor $\Gamma=1/\sqrt{1-\beta^2}$, where $\Gamma\approx2-3$~ for jets \citep[e.g.,][]{wardle1997fast}. For such speeds, relativistic beaming (or Doppler boosting) will become important. Consider a radiating gas cloud that is moving with relativistic speeds. It is going to appear brighter than at rest when it is moving towards us while it appears fainter when it moves away from us. For a jet with a bulk Lorentz factor $\Gamma$, the doppler boosting factor is given by:
\begin{equation}
    \delta = \frac{1}{\Gamma(1-\beta\cos{\theta})}
\end{equation}
where $\theta$ is the angle at which the velocity of the jet makes makes with our line of sight. 
Also, a photon of energy $E^{em}$ in the rest frame of the jet is related to the received energy $E^{rec}$ by $E^{rec}= \delta E^{em}$. 
 In other words, the energy of the photon will be \textit{redshifted} as the jet recedes from us while it is \textit{blueshifted} as the jet approaches us. 
 %The flux emitted in the frame of the jet is related to the flux received (both at the same receiving frequency) by 
 %\begin{equation}\label{eq:flux}
 %\large
 %   F^{rec}_{\nu}=F^{em}_{\nu}\delta^{3} 
%\end{equation}
When the flux density follows a powerlaw ($F_\nu\propto\nu^{-\alpha}$), the received flux would be related to the flux in the emitting frame, $F^{em}_{{\nu}}$, evaluated at the observed frequency is given by
\begin{equation}\label{eq:flux}
    F^{rec}_{\nu}=F^{em}_{{\nu}}\delta^{3+\alpha}
\end{equation}
%Note that Eq.~\ref{eq:flux_powerlaw} compares the flux density in both the emitting frame and the receiving frame at the \textit{same} frequency, while Eq.~\ref{eq:flux} does it in frequencies of their respective frames.
Also, the bulk motion of the jet tends to focus most of its emission in a narrow cone of half opening angle of $\sim1/\Gamma$. This, combined with \textit{boosting}, alters the true appearance of the jets that we observe. For example, M87, shown in Fig.~\ref{fig:M87_showpiece}, lacks a counter jet since its emission is beamed away from us.

The first confirmation that the jets are relativistic came from the time-series observations of the jets at parsec scales. These observations require extremely high angular resolution (milliarcseconds) and which is achieved by using the Very Long Baseline Interferometry (VLBI) technique \citep{cohen1971small}. The observed speeds sometime appear larger than the speed of light. Due to relativistic motion, when a radiating source moves with a relativistic speed along a line that is aligned close to our line of sight, it nearly catches up its own radiation. This can make the source appear as if its apparent transverse motion is greater than the speed of light: a \textit{superluminal} motion. The apparent transverse speed is given by
\begin{equation}
    \beta_{app}=\frac{\beta\sin{\theta}}{1-\beta\cos{\theta}}
\end{equation}
where $\theta$~is the angle that the velocity makes with our line of sight.
For $\beta\approx1$, and when the jet is inclined close to our line of sight, the apparent speed can exceed the speed of light. For a given $\Gamma$, maximum apparent motion is observed at a \textit{critical angle}, $\cos{\theta_c}=\beta$. The critical angle also yields a lower limit on $\Gamma$, which is given by
\begin{equation}
    \Gamma\geq\sqrt{\beta_{app}^2+1}
\end{equation}
Conversely, by allowing $\beta\to1$, which is the maximum allowed intrinsic speed, we can obtain the maximum viewing angle:
\begin{equation}
    \cos{\theta_{max}}=\frac{\beta_{app}^2-1}{\beta_{app}^2+1}
\end{equation}
For example, if a jet exhibits an apparent transverse motion of $\beta_{app}=20$, it implies $\Gamma\geq20.02$ with a maximum viewing angle $\theta\leq5.7^\circ$.
Furthermore, thanks to the radio telescopes such as the Very Long Baseline Array (VLBA), we have identified hundreds of jets that exhibit superluminal motions \citep[e.g.,][]{2018ApJS..234...12L}; they have significantly advanced our knowledge about jet physics.
\subsection{The Central Engine \label{sec:engine}}
The most challenging properties of AGN during the early period  of their discovery were their
high luminosity ($L\sim10^{46}$ erg/s) and short variability time scales ($\sim$few days). Their maximum luminosity is about a million times of our entire galaxy, and it could not be due to a single massive star. Their short variability time scales imply a tiny emitting region as any significant change in the luminosity can only occur on times greater than the light-crossing of the emitting region. This led to the idea that accretion on to SMBH powers the radiative output of quasars \citep{salpeter1964accretion,zel1964estimating,lynden1969galactic}, which is the currently accepted model.

\subsubsection{Evidence for SMBHs in AGN}
Although the idea of an SMBH at the center of an AGN quickly gained acceptance by the community, it took a long time to infer the presence of black holes. So indirect methods that use the effects of black holes on their surroundings became the key. Monitoring star trajectories around a putative location of the black hole can reveal its presence. However, these stars orbit on the order of a  parsec in distance from the black hole. Hence, this method could only be applied to a few nearby galaxies, the observations that current high angular resolution telescopes such as HST can support. Indeed, the properties of the SMBH at our galaxy's center remain the most precise ever to be measured using this method \citep{2008ApJ...689.1044G}.

Reverberation mapping is another important technique that was developed in the 1990s (see \citet{Peterson2004} for a review) to measure the properties of SMBHs. In a simple scenario, the continuum from the accretion disk (see section~\ref{subsec:accretion}) ionized the gas clouds in the BLR that, in turn, emit different emission lines at different distances from the black hole. A change in brightness of the continumm would lead to a change in the brightness of the emission lines but with an associated time lag. This time lag, combined with the light-crossing time, places an upper limit on the size of the BLR and also determines the mass of the black hole \citep{Peterson2004}. Another key insight was gained in the 1990s, where it was found that the mass of a galaxy's SMBH correlates with the surrounding spheroidal distribution of stars known as the galactic bulge \citep{magorrian1998demography}. The observed stellar velocity dispersion at the center of a galaxy is known to scale with its bulge mass. This dispersion is currently used to estimate the mass of the SMBHs and is known as the $M-\sigma$~relation
\citep{ferrarese2000fundamental,gebhardt2000relationship}. 
In 2019, the first-ever image of the shadow of a SMBH at the center of M87 was taken by the Event Horizon Telescope (EHT) collaboration, which constitutes a direct observational evidence for the presence of an SMBH \citep{collaborat2019first}.
%[CITATION]
\subsubsection{Disk Accretion and the Eddington Limit\label{subsec:accretion}}
There is strong evidence that accretion of matter on to a black hole is what powers the radiative output of AGNs. The accretion process is extremely efficient at converting the gravitational potential energy into radiation. The first type of accretion that was studied is a ``spherical" type and is commonly called \textit{Bondi} accretion \citep{bondi1952spherically}. Bondi is only a good approximation for isolated compact objects and is unlikely to be the main process behind AGN activity. It is because of the lack of angular momentum in spherical accretion: the particles do not have time to radiate their thermal energy before they fall in to the black hole. However, if the in-falling matter forms a disk while possessing angular momentum, its radiative efficiency can be significantly higher. In the case of objects like AGN, an accretion disk will be formed. For a black hole that accretes matter at a rate of $\dot{M}$~and converts it into radiation with an efficiency $\eta$, the bolometric luminosity of the accretion disk is given by $L_{acc}=\eta\dot{M}c^2$. The efficiency parameter $\eta$ can range between 0.06 and 0.3 and depends on the spin of the black hole, the type of accretion, and other parameters that are far from a clear understanding \citep{raimundo2009eddington}.
%[NEWER CITATION]  

As pointed out by Arthur Eddington in the 1920s, for a spherical mode of accretion, the radiation from the inner parts of the in-falling matter exerts pressure on the outer parts, thereby limiting the accretion%[CAREFUL.: EDDINGTON CONSIDERED SPHERICAL ACCRETION]
. The limiting luminosity depends on the type of the in-falling matter, which is ionized hydrogen, in most astrophysical environments, and the mass of the black hole. It is called the \textit{Eddington} luminosity. For a black hole that predominantly accretes ionized hydrogen, the Eddington luminosity (in ergs s$^{-1}$) is given by
\begin{equation}
    L_{edd}\simeq1.3\times10^{38}\frac{M}{M_\odot}{\rm erg/
    s} 
    \simeq 3\times10^4\frac{M}{M_\odot}L_\odot
\end{equation}
where $M_\odot$ and $L_\odot$ are mass and luminosity of the sun, respectively. A bright quasar has a luminosity of $\sim10^{13}L_\odot$, which makes the mass of its black hole to exceed $\sim3\times10^8M_\odot$ thereby indicating the massive nature of black holes that power AGN. However, $L_{edd}$ will be lower if the ionizing plasma contains heavier elements that increase the absorption of photons. Also, if there is a directional asymmetry in the distribution of the in-falling matter, it will accordingly decrease $L_{edd}$ \citep{frank2002accretion}. Hence, these two factors, which are not well understood, may introduce uncertainty in the Eddington limit.
%[NOT CLEAR].
Although, in principle, Bondi accretion may take place in nearby non-active galaxies, the observed X-ray flux is orders of magnitudes smaller than what is expected \citep[e.g.,][]{wong2011resolving}, indicating the presence of an even more inefficient accretion process.
%[WHO CLAIMS THAT?]



\subsubsection{Raising the jet}
How the jets are launched by the black hole or what is the matter content of the jet plasma is currently unknown. Over the years, many published jet launching mechanisms have commonly assumed that the jets acquire their mass from a certain fraction of the accreting matter. However, they differ in whether or not the magnetic field is primarily responsible for ejecting the jet.

Pure hydrodynamic models based on winds from accretion disks such as those powered by binary black holes \citep[e.g.,][]{1973A&A....24..337S} have limited capabilities. They cannot accelerate the jets to relativistic speeds, such as those seen in the VLBI observations. However, low-efficiency accretion flows can produce adequately relativistic outflows with reasonable collimation \citep{rees1982ion,das1998observational}. These models make some critical assumptions, and their applicability is yet to be verified.

The current belief is that magnetic fields play an important role in launching the jet and collimating them. One fact that supports this view is that many jets emit radio emission via the synchrotron mechanism, which requires a magnetic field. Also, magnetic fields can naturally produce relativistic speeds and collimated outflows with large kinetic energies \citep[e.g.,][]{heinz2000jet}. These factors thereby make magnetic field models superior to pure hydrodynamic models. 

The magnetic field models generally come in two flavors. They either harvest energy and angular momentum from the accretion disk \citep[e.g.,][BP]{blandford1982hydromagnetic}, or from the spin of the black hole that threads the magnetic field at its horizon \citep[e.g.,][BZ]{blandford1977electromagnetic}. For the BP mechanism, one would expect a tight correlation between the accretion power and jet power, which is observed in some Galactic systems \citep[e.g.,][]{willott1999emission}. However, there is also evidence for correlation between black hole-spin and  jet power \citep[e.g.,][]{mcclintock2013black} and recent 3D relativistic magnetohydrodynamic simulations of jets support this scenario \citep{tchekhovskoy2011efficient}.
%Nevertheless, gaining deep insights into the physics of the jet launching would likely require more detailed large scale numerical simulations. 
See \citet{mckinney2006general} and \citet{marti2019numerical} for reviews on jet simulations. 

%It is important to note that most of the simulations reproduce the observed properties of jets. However, they might not represent the actual conditions that exist in a jet. [NOT REALLY TRUE. MOST SIMS DO NOT HAVE PARTICLE ACCELERTION COUPLED TO MHD AND THEY DO NOT HAVE PROPER RADIATIVE PROCESSES AND COUPLING TO THE JET FLOW]

An important unknown is the particle composition of the jet. The material that transports energy and momentum from the parsec scales of the AGN out to kpc and Mpc scales is still uncertain. Note that this material may not necessarily be the radiation-emitting plasma. For example, electrons with energies $\gamma\geq2000$\footnote{$gamma$=\frac{1}{\sqrt(1-(v/c)^2} is the Lorentz factor} would not survive until the tip of the jet  \citep{2007RMxAC..27..188H} and hence cannot be the energy carriers of a jet. A few possibilities for the jet material include cold (thermal) or hot (relativistic) protons, cold electrons/positron, and Poynting flux (energy stored in magnetic fields).  Sometimes, neutrons are also suggested as a candidate jet material \citep{dermer2004nonthermal}. The latter idea did not gain much traction mainly because of the difficulties with jet formation and jet bending on kpc scales. \citet{sikora2000pair} argue that protons are the main energy carriers of the jet but electrons (and positrons) outnumber them. \citep{sikora2005quasar} extend this argument by suggesting that the Poynting flux dominates the jet initially and the particles would dominate later with protons mainly carrying the energy of the jet.

The location where the jet reaches its peak Lorentz factor (speed) is also uncertain. It could happen extremely close to the black hole, or the jet could be progressively accelerated while propagating away from the black hole  \citep[see~][]{meier2003theory}. The latter mode is commonly adopted because if the jet is accelerated close to the black hole, it will experience significant energy loss by inverse-Compton scattering (see section \ref{subsec:ic}) of the ambient photon field, a phenomenon that is commonly known as the ``Compton drag" effect.
More detailed observations, theoretical studies, and numerical simulations are necessary to resolve these questions.












